\documentclass[twocolumn]{article} % 启用双栏排版
\usepackage{arxiv} % 特定于arxiv的样式包,用于格式设置
% packages
\usepackage[utf8]{inputenc} % 允许utf-8输入
\usepackage{fontspec}
\usepackage{url}            % 用于类型设置URL
\usepackage{booktabs}       % 创建专业质量的表格
\usepackage{amsfonts}       % 黑板数学符号
\usepackage{nicefrac}       % 紧凑的分数符号
\usepackage{microtype}      % 微排版
\usepackage{lipsum}         % 生成填充文本
\usepackage{graphicx}       % 图形包
\usepackage{doi}            % 处理DOI
\usepackage{titlesec}       % 调整节标题的间距和格式
\usepackage{setspace}
\setstretch{1.1}
\usepackage[backend=biber, natbib=true]{biblatex} % 使用biblatex处理参考文献,指定使用biber作为后端和IEEE样式
\usepackage{xcolor}
\usepackage{amsmath}
\usepackage{tabularx}
\usepackage{colortbl}

\usepackage{adjustbox}


\usepackage{libertinus}


\usepackage{mathtools}
% 设置所有显示公式自动编号
\mathtoolsset{showonlyrefs=false}

% Pandoc 3.2.1 introduced new LaTex macro \pandocbounded: https://github.com/Wandmalfarbe/pandoc-latex-template/issues/391
\newcommand{\pandocbounded}[1]{\begin{adjustbox}{max width=\textwidth,max height=\textheight,keepaspectratio}#1\end{adjustbox}}

% caption front
\usepackage{caption}
\captionsetup{font=small,labelfont=bf}


% 配置链接颜色
\definecolor{authorcolor}{HTML}{800080} % 淡紫色
\definecolor{linkcolor}{HTML}{A6CE39}   % 绿色

% 标题设置
\title{\large\bfseries\textit{GSplatLoc} : Ultra-Precise Camera
Localization via 3D Gaussian Splatting}


\usepackage{titling}
\usepackage{hyperref}

\usepackage{footnote}

% 自定义作者信息命令
\newcommand{\authorinfo}[5]{%
  \begin{tabular}[t]{c}
    \href{#1}{\includegraphics[scale=0.06]{orcid.pdf}\hspace{1mm}\textbf{#2}}#5\\
    \texttt{#3}\\
    #4
  \end{tabular}%
}

% 重定义 \and 命令以添加适当的间距
\renewcommand{\and}{\hspace{2em}}

% 作者设置
\author{%
\authorinfo{https://orcid.org/0009-0008-5460-325X}{Atticus
Zhou}{zhouge1831@gmail.com}{Southeast University Chengxian
College\\Nanjing,
China}{}\and\authorinfo{https://orcid.org/0000-0000-0000-0000}{Jane
Doe}{jane.doe@another.edu}{Southeast University Chengxian
College\\Nanjing, China}{\textsuperscript{*}}%
}

\usepackage{dblfnote}  % 添加这行以改善双栏模式下的脚注

% 重新定义 \footnotetext 命令
\makeatletter
\def\footnotetext{%
  \@ifnextchar [\@xfootnotenext
    {\@xfootnotenext[\@mpfn]}}
\def\@xfootnotenext[#1]{%
  \begingroup
  \csname c@#1\endcsname \@footnotetext
  \endgroup}
\makeatother

% 添加对应作者注释为脚注
\newcommand{\correspondingauthor}{%
  \footnotetext{\textsuperscript{*}Corresponding author}%
}

% 预定义作者格式
\preauthor{\begin{center}\large\bfseries}
\postauthor{\par\end{center}}

% PDF元数据
\hypersetup{
    pdftitle={GSplatLoc : Ultra-Precise Camera Localization via 3D
Gaussian Splatting},
    pdfauthor={Atticus Zhou, Jane Doe},
}



% % csl for pandoc --citeproc https://github.com/Zettlr/Zettlr/issues/4879
% definitions for citeproc citations
\NewDocumentCommand\citeproctext{}{}
\NewDocumentCommand\citeproc{mm}{%
\begingroup\def\citeproctext{#2}\cite{#1}\endgroup}
\makeatletter
% allow citations to break across lines
\let\@cite@ofmt\@firstofone
% avoid brackets around text for \cite:
\def\@biblabel#1{}
\def\@cite#1#2{{#1\if@tempswa , #2\fi}}
\makeatother
\newlength{\cslhangindent}
\setlength{\cslhangindent}{1.5em}
\newlength{\csllabelwidth}
\setlength{\csllabelwidth}{1.5em}
\newenvironment{CSLReferences}[2] % #1 hanging-indent, #2 entry-spacing
{\begin{list}{}{%
  \setlength{\itemindent}{0pt}
  \setlength{\leftmargin}{0pt}
  \setlength{\parsep}{0pt}
  % turn on hanging indent if param 1 is 1
  \ifodd #1
  \setlength{\leftmargin}{\cslhangindent}
  \setlength{\itemindent}{-1\cslhangindent}
  \fi
  % set entry spacing
  \setlength{\itemsep}{0.5\baselineskip}}}
{\end{list}}
\usepackage{calc}
\newcommand{\CSLBlock}[1]{\hfill\break\parbox[t]{\linewidth}{\strut\ignorespaces#1\strut}}
\newcommand{\CSLLeftMargin}[1]{\parbox[t]{\csllabelwidth}{\strut#1\strut}}
\newcommand{\CSLRightInline}[1]{\parbox[t]{\linewidth - \csllabelwidth}{\strut#1\strut}}
\newcommand{\CSLIndent}[1]{\hspace{\cslhangindent}#1}


% tightlist error using Pandoc with Markdown https://stackoverflow.com/questions/40438037/tightlist-error-using-pandoc-with-markdown
\providecommand{\tightlist}{%
  \setlength{\itemsep}{0pt}\setlength{\parskip}{0pt}}

% main
\begin{document}

\twocolumn[
\begin{@twocolumnfalse}
  \maketitle
  \correspondingauthor
  \begin{abstract}
    We present \textbf{GSplatLoc}, a camera localization method that
    leverages the differentiable rendering capabilities of 3D Gaussian
    splatting for ultra-precise pose estimation. By formulating pose
    estimation as a gradient-based optimization problem that minimizes
    discrepancies between rendered depth maps from a pre-existing 3D
    Gaussian scene and observed depth images, GSplatLoc achieves
    translational errors within \textbf{0.01\,cm} and near-zero
    rotational errors on the Replica dataset---significantly
    outperforming existing methods. Evaluations on the Replica and TUM
    RGB-D datasets demonstrate the method's robustness in challenging
    indoor environments with complex camera motions. GSplatLoc sets a
    new benchmark for localization in dense mapping, with important
    implications for applications requiring accurate real-time
    localization, such as robotics and augmented reality.
  \end{abstract}
  \vspace{1cm}
\end{@twocolumnfalse}
]

\section{Introduction}\label{introduction}

Visual localization, specifically the task of estimating camera position
and orientation (pose estimation) for a given image within a known
scene, is a fundamental challenge in computer vision. Accurate pose
estimation is crucial for applications like autonomous robotics (e.g.,
self-driving cars), as well as Augmented and Virtual Reality systems.
Although Visual Simultaneous Localization and Mapping (Visual SLAM)
combines both mapping and pose estimation, this paper focuses
specifically on the localization component, which is essential for
real-time tracking in dynamic environments.

Traditional SLAM systems {[}1{]} have demonstrated accurate pose
estimation across diverse environments. However, their underlying 3D
representations (e.g., point clouds, meshes, and surfels) exhibit
limitations in flexibility for tasks like photorealistic scene
exploration and fine-grained map updates. Recent methods utilizing
Neural Radiance Fields (NeRF) {[}2{]} for surface reconstruction and
view rendering have inspired novel SLAM approaches {[}3{]}, which show
promising results in tracking and scene modeling. Despite these
advances, existing NeRF-based methods rely on computationally expensive
volume rendering pipelines, limiting their ability to perform real-time
\textbf{pose estimation} effectively.

The development of \textbf{3D Gaussian Splatting} {[}4{]} for efficient
novel view synthesis presents a promising solution to these limitations.
Its rasterization-based rendering pipeline enables faster image-level
rendering, making it more suitable for real-time applications. However,
integrating 3D Gaussian fields into SLAM systems still faces challenges,
such as overfitting to input images due to anisotropic Gaussian fields
and a lack of explicit multi-view constraints.

Current SLAM methods using 3D Gaussian Splatting, such as RTG-SLAM
{[}5{]} and GS-ICP-SLAM {[}6{]}, rely primarily on ICP-based techniques
for pose estimation. Other approaches, like Gaussian-SLAM {[}7{]}, adapt
traditional RGB-D odometry methods. While these methods have shown
potential, they often do not fully exploit the differentiable nature of
the Gaussian Splatting representation, particularly for real-time and
efficient \textbf{pose estimation}.

In this paper, we introduce \textbf{GSplatLoc}, a novel camera
localization method that leverages the differentiable properties of 3D
Gaussian Splatting specifically for efficient and accurate \textbf{pose
estimation}. Rather than addressing the full SLAM pipeline, our approach
is designed to focus solely on the localization aspect, allowing for
more efficient use of the scene representation and camera pose
estimation. By developing a fully differentiable pipeline, GSplatLoc can
be seamlessly integrated into existing Gaussian Splatting SLAM
frameworks or other deep learning tasks focused on localization.

Our main contributions are as follows:

\begin{enumerate}
\def\labelenumi{\arabic{enumi}.}
\tightlist
\item
  We present a GPU-accelerated framework for real-time camera
  localization, based on a comprehensive theoretical analysis of camera
  pose derivatives in 3D Gaussian Splatting.
\item
  We propose a novel optimization approach that focuses on camera pose
  estimation given a 3D Gaussian scene, fully exploiting the
  differentiable nature of the rendering process.
\item
  We demonstrate the effectiveness of our method through extensive
  experiments, showing competitive or superior pose estimation results
  compared to state-of-the-art SLAM approaches utilizing advanced scene
  representations.
\end{enumerate}

By focusing specifically on the challenges of localization in Gaussian
Splatting-based scenes, GSplatLoc opens new avenues for high-precision
\textbf{camera pose estimation} in complex environments. Our work
contributes to the ongoing advancement of visual localization systems,
pushing the boundaries of accuracy and real-time performance in 3D scene
understanding and navigation.

\section{Related Work}\label{related-work}

Camera localization is a fundamental problem in computer vision and
robotics, crucial for applications such as autonomous navigation,
augmented reality, and 3D reconstruction. Accurate and efficient pose
estimation remains challenging, especially in complex and dynamic
environments. In this section, we review the evolution of camera
localization methodologies, focusing on classical RGB-D localization
methods, NeRF-based approaches, and recent advancements in
Gaussian-based techniques that utilize Iterative Closest Point (ICP)
algorithms. We highlight their contributions and limitations, which
motivate the development of our proposed method.

\subsection{Classical RGB-D
Localization}\label{classical-rgb-d-localization}

Traditional RGB-D localization methods leverage both color and depth
information to estimate camera poses. These methods can be broadly
categorized into feature-based, direct, and hybrid approaches.

\textbf{Feature-Based Methods} involve extracting and matching keypoints
across frames to estimate camera motion. Notable systems such as
ORB-SLAM2 {[}8{]} and ORB-SLAM3 {[}9{]} rely on sparse feature
descriptors like ORB features. These systems have demonstrated robust
performance in various environments, benefiting from the maturity of
feature detection and matching algorithms. However, their reliance on
distinct visual features makes them less effective in textureless or
repetitive scenes. While they can utilize depth information for scale
estimation and map refinement, they primarily depend on RGB data for
pose estimation, making them susceptible to lighting changes and
appearance variations.

\textbf{Direct Methods} estimate camera motion by minimizing the
photometric error between consecutive frames, utilizing all available
pixel information. Methods such as Dense Visual Odometry (DVO) {[}1,
10{]} and DTAM{[}11{]} incorporate depth data to enhance pose estimation
accuracy. These methods can achieve high precision in well-lit, textured
environments but are sensitive to illumination changes and require good
initialization to avoid local minima. The computational cost of
processing all pixel data poses challenges for real-time applications.
Additionally, the reliance on photometric consistency makes them
vulnerable to lighting variations and dynamic scenes.

\textbf{Hybrid Approaches} combine the strengths of feature-based and
direct methods. ElasticFusion {[}12{]} integrates surfel-based mapping
with real-time camera tracking, using both photometric and geometric
information. DVO-SLAM {[}10{]} combines geometric and photometric
alignment for improved robustness. However, these methods often involve
complex pipelines and can be computationally intensive due to dense map
representations and intricate data association processes.

Despite their successes, classical methods face challenges in balancing
computational efficiency with pose estimation accuracy, particularly in
dynamic or low-texture environments. They may not fully exploit the
potential of depth information for robust pose estimation, especially
under lighting variations. Moreover, the lack of leveraging
differentiable rendering techniques limits their ability to perform
efficient gradient-based optimization for pose estimation.

\subsection{NeRF-Based Localization}\label{nerf-based-localization}

The advent of Neural Radiance Fields (NeRF) {[}2{]} has revolutionized
novel view synthesis by representing scenes as continuous volumetric
functions learned from images. NeRF has inspired new approaches to
camera localization by leveraging its differentiable rendering
capabilities.

\textbf{Pose Estimation with NeRF} involves inverting a pre-trained NeRF
model to recover camera poses by minimizing the photometric error
between rendered images and observed images. iNeRF {[}13{]} formulates
pose estimation as an optimization problem, using gradient-based methods
to refine camera parameters. While iNeRF achieves impressive accuracy,
it suffers from high computational costs due to the per-pixel ray
marching required in NeRF's volumetric rendering pipeline. This
limitation hampers its applicability in real-time localization tasks.

\textbf{Accelerated NeRF Variants} aim to address computational
inefficiency by introducing explicit data structures. Instant-NGP
{[}14{]} uses hash maps to accelerate training and rendering, achieving
interactive frame rates. PlenOctrees {[}15{]} and Plenoxels {[}16{]}
employ sparse voxel grids to represent the scene, significantly reducing
computation time. However, even with these optimizations, rendering
speeds may still not meet the demands of real-time localization in
dynamic environments.

Furthermore, NeRF-based localization methods rely heavily on photometric
consistency, making them sensitive to lighting variations, dynamic
objects, and non-Lambertian surfaces. This reliance on RGB data can
reduce robustness in real-world conditions where lighting can change
dramatically. Additionally, the extensive training required on specific
scenes limits their adaptability to new or changing environments.

\subsection{Gaussian-Based
Localization}\label{gaussian-based-localization}

Recent advancements in scene representation have introduced 3D Gaussian
splatting as an efficient alternative to NeRF. \textbf{3D Gaussian
Splatting} {[}4{]} represents scenes using a set of 3D Gaussian
primitives and employs rasterization-based rendering, offering
significant computational advantages over volumetric rendering.

\textbf{Gaussian Splatting in Localization} has been explored in methods
such as SplaTAM {[}17{]}, CG-SLAM {[}18{]}, RTG-SLAM {[}5{]}, and
GS-ICP-SLAM {[}6{]}. SplaTAM introduces a SLAM system that uses
gradient-based optimization to refine both the map and camera poses,
utilizing RGB-D data and 3D Gaussians for dense mapping. CG-SLAM focuses
on an uncertainty-aware 3D Gaussian field to improve tracking and
mapping performance, incorporating depth uncertainty modeling.

Pose estimation approaches in these methods often rely on traditional
point cloud registration techniques, such as Iterative Closest Point
(ICP) algorithms {[}19{]}. \textbf{RTG-SLAM} employs ICP for pose
estimation within a 3D Gaussian splatting framework, demonstrating
real-time performance in 3D reconstruction tasks. Similarly,
\textbf{GS-ICP-SLAM} utilizes Generalized ICP {[}20{]} for alignment,
effectively handling the variability in point cloud density and
improving robustness.

\textbf{Gaussian-SLAM} {[}7{]} adapts traditional RGB-D odometry
methods, combining colored point cloud alignment {[}21{]} with an
energy-based visual odometry approach {[}22{]}. These methods integrate
ICP-based techniques within Gaussian-based representations to estimate
camera poses.

While effective in certain scenarios, the reliance on ICP-based methods
introduces limitations. ICP algorithms require good initial alignment
and can be sensitive to local minima, often necessitating careful
initialization to ensure convergence. Additionally, ICP can be
computationally intensive, especially with large point sets, hindering
real-time performance. These methods may not fully exploit the
differentiable rendering capabilities of 3D Gaussian representations for
pose optimization.

Optimizing camera poses using depth information within a differentiable
rendering framework offers several advantages, particularly in
environments with challenging lighting or low-texture surfaces. Depth
data provides direct geometric information about the scene, which is
invariant to illumination changes, leading to more robust pose
estimation. By focusing on depth-only optimization, methods can achieve
robustness to lighting variations and improve computational efficiency
by avoiding the processing of color data.

However, existing Gaussian-based localization techniques have not fully
exploited depth-only optimization within a differentiable rendering
framework. Challenges such as sensor noise, incomplete depth data due to
occlusions, and the need for accurate initial pose estimates remain.
Furthermore, many approaches tightly couple mapping and localization,
introducing unnecessary computational overhead and complexity when the
primary goal is pose estimation.

These limitations motivate the development of our proposed method. By
leveraging the differentiable rendering capabilities of 3D Gaussian
splatting specifically for depth-only pose optimization, we aim to
overcome the challenges faced by existing methods. Our approach
eliminates reliance on photometric data, enhancing robustness to
lighting variations and reducing computational overhead. By decoupling
localization from mapping, we simplify the optimization process, making
it more suitable for real-time applications. Additionally, using
quaternions for rotation parameterization {[}23{]} and careful
initialization strategies improves the stability and convergence of the
optimization, addressing challenges associated with sensor noise and
incomplete data.

Our method fully exploits the differentiable rendering pipeline to
perform efficient gradient-based optimization for pose estimation,
setting it apart from ICP-based approaches. By focusing on depth
information and leveraging the strengths of 3D Gaussian splatting, we
provide a robust and computationally efficient solution for camera
localization in complex environments.

\section{Method}\label{method}

\begin{figure*}[!t]
\vspace{-2cm}
\centering
\resizebox{\textwidth}{!}{\includegraphics{Flowcharts.pdf}}
\caption{We propose **GSplatLoc**, a novel camera localization method that leverages the differentiable rendering capabilities of 3D Gaussian splatting for efficient and accurate pose estimation.}
\label{fig:cross-column-image}
\end{figure*}

\textbf{Overview.} We propose \textbf{GSplatLoc}, a novel camera
localization method that leverages the differentiable rendering
capabilities of 3D Gaussian splatting for efficient and accurate pose
estimation. By formulating pose estimation as a gradient-based
optimization problem within a fully differentiable framework, GSplatLoc
enables direct optimization of camera poses using depth information
rendered from a pre-existing 3D Gaussian scene representation. This
approach allows us to achieve high-precision localization suitable for
real-time applications.

\textbf{Motivation.} Traditional SLAM systems that use point clouds,
meshes, or surfels for 3D representation often face limitations in
rendering quality and computational efficiency, hindering their ability
to provide photorealistic scene exploration and fine-grained map
updates. Neural Radiance Fields (NeRF) {[}2{]} have demonstrated
exceptional rendering quality but suffer from computational
inefficiencies due to per-pixel ray marching in volume rendering, making
real-time applications challenging.

The recent development of \textbf{3D Gaussian Splatting} {[}4{]} offers
a promising alternative by employing a rasterization-based rendering
pipeline. In this method, scenes are represented using a set of 3D
Gaussians, which can be efficiently projected onto the image plane and
rasterized to produce high-quality renderings at interactive frame
rates. The differentiable nature of this rendering process enables
gradient computation with respect to both the scene parameters and the
camera pose.

By leveraging these properties, we aim to develop a localization method
that fully utilizes the differentiable rendering capabilities of 3D
Gaussian splatting. Our approach focuses on optimizing the camera pose
by minimizing the difference between the rendered depth map and the
observed query depth image, thus enabling accurate and efficient pose
estimation suitable for real-time SLAM systems.

\textbf{Problem Formulation.} Our objective is to estimate the 6-DoF
pose \((\mathbf{R}, \mathbf{t}) \in SE(3)\) of a query depth image
\(D_q\), where \(\mathbf{R}\) is the rotation matrix and \(\mathbf{t}\)
is the translation vector in the camera coordinate system. Given a 3D
representation of the environment in the form of 3D Gaussians, let
\(\mathcal{G} = \{G_i\}_{i=1}^N\) denote a set of \(N\) Gaussians, and
posed reference depth images \(\{D_k\}\), which together constitute the
reference data.

In the following sections, we detail the components of our method,
including the scene representation, the differentiable depth rendering
process, the formulation of the optimization problem, and the overall
pipeline for camera localization.

\subsection{Scene Representation}\label{scene-representation}

Building upon the Gaussian splatting method {[}4{]}, we adapt the scene
representation to focus on the differentiable depth rendering process,
which is crucial for our localization task. Our approach utilizes the
efficiency and quality of Gaussian splatting while tailoring it
specifically for depth-based localization.

\textbf{3D Gaussians.} Each Gaussian \(G_i\) is characterized by its 3D
mean \(\boldsymbol{\mu}_i \in \mathbb{R}^3\), 3D covariance matrix
\(\boldsymbol{\Sigma}_i \in \mathbb{R}^{3\times3}\), opacity
\(o_i \in \mathbb{R}\), and scale \(\mathbf{s}_i \in \mathbb{R}^3\). To
represent the orientation of each Gaussian, we use a rotation quaternion
\(\mathbf{q}_i \in \mathbb{R}^4\).

The 3D covariance matrix \(\boldsymbol{\Sigma}_i\) is parameterized
using \(\mathbf{s}_i\) and \(\mathbf{q}_i\):

\[
\boldsymbol{\Sigma}_i = \mathbf{R}(\mathbf{q}_i) \mathbf{S}(\mathbf{s}_i) \mathbf{S}(\mathbf{s}_i)^\top \mathbf{R}(\mathbf{q}_i)^\top
\]

where \(\mathbf{R}(\mathbf{q}_i)\) is the rotation matrix derived from
\(\mathbf{q}_i\), and
\(\mathbf{S}(\mathbf{s}_i) = \text{diag}(\mathbf{s}_i)\) is a diagonal
matrix of scales.

\textbf{Projecting 3D to 2D.} For the projection of 3D Gaussians onto
the 2D image plane, we follow the approach described by {[}4{]}. The 3D
mean \(\boldsymbol{\mu}_i\) is first transformed into the camera
coordinate frame using the world-to-camera transformation
\(\mathbf{T}_{wc} \in SE(3)\). Then, it is projected using the
projection matrix \(\mathbf{P} \in \mathbb{R}^{4 \times 4}\) and mapped
to pixel coordinates via the function
\(\pi: \mathbb{R}^4 \rightarrow \mathbb{R}^2\):

\[
\boldsymbol{\mu}_{I,i} = \pi\left( \mathbf{P} \mathbf{T}_{wc} \boldsymbol{\mu}_{i,\text{homogeneous}} \right)
\]

Similarly, the 2D covariance
\(\boldsymbol{\Sigma}_{I,i} \in \mathbb{R}^{2\times2}\) of the projected
Gaussian is obtained by transforming the 3D covariance
\(\boldsymbol{\Sigma}_i\) into the image plane: \[
\boldsymbol{\Sigma}_{I,i} = \mathbf{J} \mathbf{R}_{wc} \boldsymbol{\Sigma}_i \mathbf{R}_{wc}^\top \mathbf{J}^\top
\]

where \(\mathbf{R}_{wc}\) represents the rotation component of
\(\mathbf{T}_{wc}\), and \(\mathbf{J}\) is the Jacobian of the
projection function, accounting for the affine transformation from 3D to
2D as described by {[}24{]}.

\subsection{Depth Rendering}\label{depth-rendering}

We implement a differentiable depth rendering process, which is crucial
for our localization method as it allows for gradient computation
throughout the rendering pipeline. This differentiability enables us to
optimize camera poses directly based on rendered depth maps.

\textbf{Compositing Depth.} For depth map generation, we employ a
front-to-back compositing scheme, which allows for accurate depth
estimation and proper handling of occlusions. Let \(d_n\) denote the
depth value of the \(n\)-th Gaussian, corresponding to the z-coordinate
of its mean in the camera coordinate system. The depth at pixel
\(\mathbf{p}\), denoted \(D(\mathbf{p})\), is computed as {[}4{]}:

\[D(\mathbf{p}) = \sum_{n \leq N} d_n \cdot \alpha_n \cdot T_n,\]

where \(T_n\) is the cumulative transparency up to the \((n-1)\)-th
Gaussian:

\[T_n = \prod_{m < n} (1 - \alpha_m).\]

In this formulation, \(\alpha_n\) represents the opacity contribution of
the \(n\)-th Gaussian at pixel \(\mathbf{p}\), calculated as:

\[\alpha_n = o_n \cdot \exp(-\sigma_n),\]

with

\[\sigma_n = \frac{1}{2} \boldsymbol{\Delta}_n^\top \boldsymbol{\Sigma}_I^{-1} \boldsymbol{\Delta}_n.\]

Here, \(\boldsymbol{\Delta}_n\) is the offset between the pixel center
and the center of the 2D Gaussian \(\boldsymbol{\mu}_{I,i}\), and
\(\boldsymbol{\Sigma}_I\) is the 2D covariance of the projected
Gaussian. The opacity parameter \(o_n\) controls the overall opacity of
the Gaussian.

\textbf{Normalization of Depth.} To ensure consistent depth
representation across the image, we normalize the accumulated depth
values. We first compute the total accumulated opacity at each pixel
\(\mathbf{p}\): \[
\alpha(\mathbf{p}) = \sum_{n \leq N} \alpha_n \cdot T_n
\]

The normalized depth at pixel \(\mathbf{p}\) is then defined as:

\[
\text{Norm}_D(\mathbf{p}) = \frac{D(\mathbf{p})}{\alpha(\mathbf{p})}
\]

This normalization ensures that the depth values are properly scaled,
making them comparable across different regions of the image, even when
the density of Gaussians varies.

The differentiable nature of this depth rendering process is key to our
localization method. It allows us to compute gradients with respect to
the Gaussian parameters and camera pose, enabling direct optimization
based on the rendered depth maps. This differentiability facilitates
efficient gradient-based optimization, forming the foundation for our
subsequent localization algorithm.

\subsection{Localization as Image
Alignment}\label{localization-as-image-alignment}

Assuming we have an existing map represented by a set of 3D Gaussians,
our localization task focuses on estimating the 6-DoF pose of a query
depth image \(D_q\) within this map. This process essentially becomes an
image alignment problem between the rendered depth map from our Gaussian
representation and the query depth image.

\textbf{Rotating with Quaternions.} We parameterize the camera pose
using a quaternion \(\mathbf{q}_{cw}\) for rotation and a vector
\(\mathbf{t}_{cw}\) for translation. This choice of parameterization is
particularly advantageous in our differential rendering context.
Quaternions provide a continuous and singularity-free representation of
rotation, which is crucial for gradient-based optimization. Moreover,
their compact four-parameter form aligns well with our differentiable
rendering pipeline, allowing for efficient computation of gradients with
respect to rotation parameters.

\textbf{Loss Function.} Our optimization strategy leverages the
differentiable nature of our depth rendering process. We define a loss
function that incorporates both depth accuracy and edge alignment:

\[
\mathcal{L} = \lambda_1 \mathcal{L}_d + \lambda_2 \mathcal{L}_c
\]

where \(\lambda_1\) and \(\lambda_2\) are weighting factors (typically
set to 0.8 and 0.2, respectively) that balance the contributions of the
depth and contour losses. The depth loss \(\mathcal{L}_d\) measures the
L1 difference between the rendered depth map and the observed depth
image:

\[
\mathcal{L}_d = \sum_{i \in \mathcal{M}} \left| D_i^{\text{rendered}} - D_i^{\text{observed}} \right|
\]

The contour loss \(\mathcal{L}_c\) focuses on aligning the depth
gradients (edges) between the rendered and observed depth images:

\[
\mathcal{L}_c = \sum_{j \in \mathcal{M}} \left| \nabla D_j^{\text{rendered}} - \nabla D_j^{\text{observed}} \right|
\]

Here, \(\nabla D\) represents the gradient of the depth image, computed
using the Sobel operator {[}25{]}, and \(\mathcal{M}\) is the mask of
valid pixels determined by the rendered alpha mask.

The contour loss \(\mathcal{L}_{c}\) serves several crucial purposes. It
ensures that depth discontinuities in the rendered image align well with
those in the observed depth image, thereby improving the overall
accuracy of the pose estimation. By explicitly considering edge
information, we preserve important structural features of the scene
during optimization. Furthermore, the contour loss is less sensitive to
absolute depth values and more focused on relative depth changes, making
it robust to global depth scale differences.

The optimization objective can be formulated as:

\[
\min_{\mathbf{q}_{cw}, \mathbf{t}_{cw}} \mathcal{L} + \lambda_q \|\mathbf{q}_{cw}\|_2^2 + \lambda_t \|\mathbf{t}_{cw}\|_2^2
\]

where \(\lambda_q\) and \(\lambda_t\) are regularization coefficients
for the quaternion and translation parameters, respectively.

\textbf{Masking Uncertainty.} The rendered alpha mask plays a crucial
role in our optimization process. It effectively captures the epistemic
uncertainty of our map, allowing us to focus the optimization on
well-represented parts of the scene. By utilizing this mask, we avoid
optimizing based on unreliable or non-existent data, which could
otherwise lead to erroneous pose estimates.

\textbf{Optimization Parameters.} We perform the optimization using the
Adam optimizer, with distinct learning rates and weight decay values for
the rotation and translation parameters. Specifically, we set the
learning rate for quaternion optimization to \(5 \times 10^{-4}\) and
for translation optimization to \(10^{-3}\), based on empirical tuning.
Weight decay, set to \(10^{-3}\) for both parameters, acts as a
regularization term to prevent overfitting. These settings balance the
trade-off between convergence speed and optimization stability.

\subsection{Pipeline}\label{pipeline}

Our GSplatLoc method streamlines the localization process by utilizing
only the posed reference depth images \(\{D_k\}\) and the query depth
image \(D_q\). The differentiable rendering of 3D Gaussians enables
efficient and smooth convergence during optimization.

\textbf{Evaluation Scene.} For consistent evaluation, we initialize the
3D Gaussians from point clouds derived from the posed reference depth
images \(\{D_k\}\). Each point in the point cloud corresponds to the
mean \(\boldsymbol{\mu}_i\) of a Gaussian \(G_i\). After filtering out
outliers, we set the opacity \(o_i = 1\) for all Gaussians to ensure
full contribution in rendering. The scale \(\mathbf{s}_i\) is
initialized isotropically based on the local point density:

\[
\mathbf{s}_i = (\sigma_i, \sigma_i, \sigma_i), \quad \text{with } \sigma_i = \sqrt{\frac{1}{3}\sum_{j=1}^3 d_{ij}^2}
\]

where \(d_{ij}\) is the distance to the \(j\)-th nearest neighbor of
point \(i\), computed using k-nearest neighbors (with \(k=4\)). This
initialization balances the representation of local geometry. The
rotation quaternion \(\mathbf{q}_i\) is initially set to
\((1, 0, 0, 0)\) for all Gaussians, corresponding to no rotation.

To further enhance optimization stability, we apply standard Principal
Component Analysis (PCA) to align the principal axes of the point cloud
with the coordinate axes. By centering the point cloud at its mean and
aligning its principal axes, we normalize the overall scene orientation.
This provides a more uniform starting point for optimization across
diverse datasets, significantly improving the stability of the loss
reduction during optimization and facilitating the attainment of lower
final loss values, especially in the depth loss component of our
objective function.

\textbf{Optimization.} We employ the Adam optimizer for optimizing both
the quaternion and translation parameters, using the distinct learning
rates and weight decay values as previously described. The optimization
process greatly benefits from the real-time rendering capabilities of 3D
Gaussian splatting. Since rendering is extremely fast, each iteration of
the optimizer is limited mainly by the rendering speed, allowing for
rapid convergence of our pose estimation algorithm and making it
suitable for real-time applications.

\textbf{Convergence.} To determine convergence, we implement an early
stopping mechanism based on the stabilization of the total loss. Our
experiments show that the total loss usually stabilizes after
approximately 100 iterations. We employ a patience parameter: after 100
iterations, if the total loss does not decrease for a predefined number
of consecutive iterations, the optimization loop is terminated. We then
select the pose estimate corresponding to the minimum total loss as the
optimal pose.

In summary, our pipeline effectively combines the efficiency of Gaussian
splatting with a robust optimization strategy, resulting in a fast and
accurate camera localization method suitable for real-time applications.

\section{Evaluation}\label{evaluation}

We conducted extensive experiments to evaluate the performance of our
proposed method, \textbf{GSplatLoc}, in comparison with state-of-the-art
SLAM systems that utilize advanced scene representations. The evaluation
focuses on assessing the accuracy of camera pose estimation in
challenging indoor environments, emphasizing both the translational and
rotational components of the estimated poses.

\subsection{Experimental Setup}\label{experimental-setup}

\textbf{Implementation Details.} Our localization pipeline was
implemented on a system equipped with an Intel Core i7-13620H CPU,
16\,GB of RAM, and an NVIDIA RTX 4060 GPU with 8\,GB of memory. The
algorithm was developed using Python and PyTorch, utilizing custom CUDA
kernels to accelerate the rasterization and backpropagation processes
inherent in our differentiable rendering approach. This setup ensures
that our method achieves real-time performance, which is crucial for
practical applications in SLAM systems.

\textbf{Datasets.} We evaluated our method on two widely recognized
datasets for SLAM benchmarking: the \textbf{Replica} dataset {[}26{]}
and the \textbf{TUM RGB-D} dataset {[}27{]}. The Replica dataset
provides high-fidelity synthetic indoor environments, ideal for
controlled evaluations of localization algorithms. We utilized data
collected by Sucar et al. {[}28{]}, which includes trajectories from an
RGB-D sensor with ground-truth poses. The TUM RGB-D dataset offers
real-world sequences captured in various indoor settings, providing a
diverse range of scenarios to test the robustness of our method.

\textbf{Metrics.} Localization accuracy was assessed using two standard
metrics: the \textbf{Absolute Trajectory Error (ATE RMSE)}, measured in
centimeters, and the \textbf{Absolute Angular Error (AAE RMSE)},
measured in degrees. The ATE RMSE quantifies the root mean square error
between the estimated and ground-truth camera positions, while the AAE
RMSE measures the accuracy of the estimated camera orientations.

\textbf{Baselines.}~To provide a comprehensive comparison, we evaluated
our method against several state-of-the-art SLAM systems that leverage
advanced scene representations. Specifically, we compared against
RTG-SLAM (ICP) {[}5{]}, which utilizes Iterative Closest Point (ICP) for
pose estimation within a 3D Gaussian splatting framework. We also
included GS-ICP-SLAM (GICP) {[}6{]}, which employs Generalized ICP for
alignment in a Gaussian-based representation. Additionally, we
considered Gaussian-SLAM {[}7{]}, evaluating both its PLANE ICP and
HYBRID variants, which adapt traditional RGB-D odometry methods by
incorporating plane-based ICP and a hybrid approach combining
photometric and geometric information. These baselines were selected
because they represent the current state of the art in SLAM systems
utilizing advanced scene representations and focus on the localization
component, which aligns with the scope of our work.

\subsection{Localization Evaluation}\label{localization-evaluation}

We first evaluated our method on the Replica dataset, which provides a
controlled environment to assess the accuracy of pose estimation
algorithms.

\begin{table}[htbp]
\renewcommand{\thetable}{\textbf{\arabic{table}}}
\renewcommand{\tablename}{\textbf{Table}}
\centering
\caption{\textbf{Replica[26] (ATE RMSE ↓[cm]).}}
\label{table:_textbf_replica_26_a}
\begin{adjustbox}{max width=\columnwidth,max height=!,center}
\begin{tabular}{lccccccccc}
\toprule
\textbf{Methods} & \textbf{Avg.} & \textbf{R0} & \textbf{R1} & \textbf{R2} & \textbf{Of0} & \textbf{Of1} & \textbf{Of2} & \textbf{Of3} & \textbf{Of4}\\
\midrule
RTG-SLAM(ICP)[5] & \cellcolor{yellow!30}0.471 & \cellcolor{yellow!30}0.429 & \cellcolor{yellow!30}0.690 & \cellcolor{yellow!30}0.544 & \cellcolor{yellow!30}0.640 & \cellcolor{yellow!30}0.336 & \cellcolor{yellow!30}0.434 & \cellcolor{yellow!30}0.281 & \cellcolor{yellow!30}0.419\\
GS-ICP-SLAM(GICP)[6] & \cellcolor{lime!50}0.593 & \cellcolor{lime!50}0.465 & \cellcolor{lime!50}0.772 & \cellcolor{lime!50}0.723 & \cellcolor{lime!50}0.681 & \cellcolor{lime!50}0.522 & \cellcolor{lime!50}0.582 & \cellcolor{lime!50}0.438 & \cellcolor{lime!50}0.558\\
Gaussian-SLAM(PLANE ICP)[7] & 0.633 & 0.476 & 0.812 & 0.781 & 0.709 & 0.541 & 0.667 & 0.449 & 0.625\\
Gaussian-SLAM(HYBRID)[7] & 0.631 & 0.476 & 0.812 & 0.781 & 0.709 & 0.537 & 0.662 & 0.446 & 0.624\\
\midrule
\textbf{Ours} & \cellcolor{green!30}\textbf{0.009} & \cellcolor{green!30}\textbf{0.007} & \cellcolor{green!30}\textbf{0.008} & \cellcolor{green!30}\textbf{0.010} & \cellcolor{green!30}\textbf{0.009} & \cellcolor{green!30}\textbf{0.009} & \cellcolor{green!30}\textbf{0.011} & \cellcolor{green!30}\textbf{0.009} & \cellcolor{green!30}\textbf{0.011}\\
\bottomrule
\end{tabular}
\end{adjustbox}
\end{table}

\textbf{Table 1.} presents the ATE RMSE results in centimeters for
various methods across different sequences in the Replica dataset. Our
method significantly outperforms the baselines, achieving an average ATE
RMSE of \textbf{0.00925 cm}, which is an order of magnitude better than
the closest competitor. This substantial improvement demonstrates the
effectiveness of our approach in accurately estimating the camera's
position. The low translational errors indicate that our method can
precisely align the observed depth images with the rendered depth from
the 3D Gaussian scene.

\begin{table}[htbp]
\renewcommand{\thetable}{\textbf{\arabic{table}}}
\renewcommand{\tablename}{\textbf{Table}}
\centering
\caption{\textbf{Replica[26] (AAE RMSE ↓[°]).}}
\label{table:_textbf_replica_26_a}
\begin{adjustbox}{max width=\columnwidth,max height=!,center}
\begin{tabular}{lccccccccc}
\toprule
\textbf{Methods} & \textbf{Avg.} & \textbf{R0} & \textbf{R1} & \textbf{R2} & \textbf{Of0} & \textbf{Of1} & \textbf{Of2} & \textbf{Of3} & \textbf{Of4}\\
\midrule
RTG-SLAM(ICP)[5] & \cellcolor{green!30}\textbf{0.576} & \cellcolor{green!30}\textbf{0.720} & \cellcolor{green!30}\textbf{0.826} & \cellcolor{yellow!30}0.744 & \cellcolor{green!30}\textbf{0.054} & \cellcolor{green!30}\textbf{0.537} & \cellcolor{yellow!30}0.360 & \cellcolor{yellow!30}0.330 & \cellcolor{yellow!30}0.430\\
GS-ICP-SLAM(GICP)[6] & \cellcolor{lime!50}1.279 & \cellcolor{lime!50}1.659 & 1.951 & 1.607 & \cellcolor{lime!50}0.281 & \cellcolor{yellow!30}0.895 & 2.580 & 1.110 & 2.940\\
Gaussian-SLAM(PLANE ICP)[7] & 1.287 & 1.834 & \cellcolor{lime!50}1.880 & \cellcolor{lime!50}1.398 & 0.305 & 1.019 & 1.060 & 1.100 & 1.130\\
Gaussian-SLAM(HYBRID)[7] & 1.955 & 2.265 & 3.493 & 2.783 & 0.287 & \cellcolor{lime!50}0.945 & \cellcolor{lime!50}0.580 & \cellcolor{lime!50}0.720 & \cellcolor{lime!50}0.630\\
\midrule
\textbf{Ours} & \cellcolor{yellow!30}0.810 & \cellcolor{yellow!30}0.931 & \cellcolor{yellow!30}1.006 & \cellcolor{green!30}\textbf{0.666} & \cellcolor{yellow!30}0.248 & 1.197 & \cellcolor{green!30}\textbf{0.011} & \cellcolor{green!30}\textbf{0.009} & \cellcolor{green!30}\textbf{0.011}\\
\bottomrule
\end{tabular}
\end{adjustbox}
\end{table}

\textbf{Table 2.} presents the Absolute Angular Error (AAE) RMSE in
degrees for various methods on the Replica dataset. Our method achieves
a competitive average AAE RMSE of \textbf{0.80982°}, indicating superior
rotational accuracy in most sequences. In sequences with significant
rotational movements, such as Of2, Of3, and Of4, our approach
consistently outperforms the baselines. For instance, in sequence Of3,
our method achieves an AAE RMSE of \textbf{0.00930°}, compared to
\textbf{0.33000°} by RTG-SLAM and higher errors by other methods. This
exceptional performance can be attributed to the effective utilization
of the differentiable rendering pipeline and the optimization strategy
that precisely aligns the depth gradients between the rendered and
observed images.

To evaluate the robustness of our method in real-world scenarios, we
conducted experiments on the TUM RGB-D dataset, which presents
challenges such as sensor noise and dynamic environments.

\begin{table}[htbp]
\renewcommand{\thetable}{\textbf{\arabic{table}}}
\renewcommand{\tablename}{\textbf{Table}}
\centering
\caption{\textbf{TUM[27] (ATE RMSE ↓[cm]).}}
\label{table:_textbf_tum_27_ate_r}
\begin{adjustbox}{max width=\columnwidth,max height=!,center}
\begin{tabular}{lcccccc}
\toprule
\textbf{Methods} & \textbf{Avg.} & \textbf{fr1/desk} & \textbf{fr1/desk2} & \textbf{fr1/room} & \textbf{fr2/xyz} & \textbf{fr3/off.}\\
\midrule
RTG-SLAM(ICP)[5] & \cellcolor{green!30}\textbf{0.576} & \cellcolor{green!30}\textbf{0.720} & \cellcolor{green!30}\textbf{0.826} & \cellcolor{yellow!30}0.744 & \cellcolor{green!30}\textbf{0.054} & \cellcolor{green!30}\textbf{0.537}\\
GS-ICP-SLAM(GICP)[6] & \cellcolor{lime!50}1.279 & \cellcolor{lime!50}1.659 & 1.951 & 1.607 & \cellcolor{lime!50}0.281 & \cellcolor{yellow!30}0.895\\
Gaussian-SLAM(PLANE ICP)[7] & 1.287 & 1.834 & \cellcolor{lime!50}1.880 & \cellcolor{lime!50}1.398 & 0.305 & 1.019\\
Gaussian-SLAM(HYBRID)[7] & 1.955 & 2.265 & 3.493 & 2.783 & 0.287 & \cellcolor{lime!50}0.945\\
\midrule
\textbf{Ours} & \cellcolor{yellow!30}0.810 & \cellcolor{yellow!30}0.931 & \cellcolor{yellow!30}1.006 & \cellcolor{green!30}\textbf{0.666} & \cellcolor{yellow!30}0.248 & 1.197\\
\bottomrule
\end{tabular}
\end{adjustbox}
\end{table}

\textbf{Table 3.} presents the ATE RMSE in centimeters for various
methods on the TUM-RGBD dataset {[}27{]}. Our method achieves
competitive results with an average ATE RMSE of \textbf{8.0982 cm},
outperforming GS-ICP-SLAM{[}6{]} and Gaussian-SLAM{[}7{]} in most
sequences. While RTG-SLAM{[}5{]} shows lower errors in some sequences,
our method consistently provides accurate pose estimates across
different environments. The increased error compared to the Replica
dataset is expected due to the real-world challenges present in the TUM
RGB-D dataset, such as sensor noise and environmental variability.
Despite these challenges, our method demonstrates robustness and
maintains reasonable localization accuracy.

\begin{table}[htbp]
\renewcommand{\thetable}{\textbf{\arabic{table}}}
\renewcommand{\tablename}{\textbf{Table}}
\centering
\caption{\textbf{TUM[27] (AAE RMSE ↓[°]).}}
\label{table:_textbf_tum_27_aae_r}
\begin{adjustbox}{max width=\columnwidth,max height=!,center}
\begin{tabular}{lcccccc}
\toprule
\textbf{Methods} & \textbf{Avg.} & \textbf{fr1/desk} & \textbf{fr1/desk2} & \textbf{fr1/room} & \textbf{fr2/xyz} & \textbf{fr3/off.}\\
\midrule
RTG-SLAM(ICP)[5] & \cellcolor{green!30}\textbf{0.916} & \cellcolor{yellow!30}1.181 & \cellcolor{yellow!30}1.557 & \cellcolor{yellow!30}1.355 & \cellcolor{yellow!30}0.138 & \cellcolor{green!30}\textbf{0.347}\\
GS-ICP-SLAM(GICP)[6] & \cellcolor{yellow!30}0.959 & \cellcolor{lime!50}1.288 & \cellcolor{lime!50}1.618 & \cellcolor{lime!50}1.363 & \cellcolor{lime!50}0.147 & \cellcolor{lime!50}0.381\\
Gaussian-SLAM(PLANE ICP)[7] & 1.090 & 1.388 & 1.791 & 1.564 & 0.182 & 0.525\\
Gaussian-SLAM(HYBRID)[7] & 1.117 & 1.426 & 2.098 & 1.594 & \cellcolor{green!30}\textbf{0.114} & \cellcolor{yellow!30}0.355\\
\midrule
\textbf{Ours} & \cellcolor{lime!50}0.979 & \cellcolor{green!30}\textbf{1.126} & \cellcolor{green!30}\textbf{1.265} & \cellcolor{green!30}\textbf{0.907} & 0.789 & 0.808\\
\bottomrule
\end{tabular}
\end{adjustbox}
\end{table}

\textbf{Table 4.} presents the AAE RMSE results in degrees for the TUM
RGB-D dataset. Our method achieves an average AAE RMSE of
\textbf{0.97928°}, which is competitive with the other methods. In
sequences such as fr1/room, our method demonstrates superior rotational
accuracy with an AAE RMSE of \textbf{0.90722°}, compared to higher
errors by the baselines. The slightly higher rotational errors in the
TUM RGB-D dataset, compared to the Replica dataset, can be attributed to
the complexities of real-world data, including sensor inaccuracies and
dynamic elements in the environment. Nonetheless, our method maintains
reliable performance across various sequences.

\subsection{Discussion}\label{discussion}

The experimental results indicate that our method consistently achieves
high localization accuracy, particularly in terms of translational
error, where we significantly outperform existing approaches on the
Replica dataset. The rotational accuracy of our method is also
competitive, often surpassing other methods in challenging sequences.
These outcomes demonstrate the effectiveness of our approach in
leveraging the differentiable rendering capabilities of 3D Gaussian
splatting for pose estimation.

Several factors contribute to the superior performance of our method. By
utilizing a fully differentiable depth rendering process, our method
allows for efficient gradient-based optimization of camera poses,
leading to precise alignment between the rendered and observed depth
images. The combination of depth loss and contour loss in our
optimization objective enables the method to capture both absolute depth
differences and structural features, enhancing the robustness and
accuracy of pose estimation. Additionally, employing quaternions for
rotation representation provides a continuous and singularity-free
parameter space, improving the stability and convergence of the
optimization process.

While our method shows excellent performance on the Replica dataset, the
increased errors on the TUM RGB-D dataset highlight areas for potential
improvement. Real-world datasets introduce challenges such as sensor
noise, dynamic objects, and incomplete depth data due to occlusions.
Addressing these challenges in future work could further enhance the
robustness of our method.

\subsection{Limitations}\label{limitations}

Despite the promising results, our method has certain limitations. The
reliance on accurate depth data means that performance may degrade in
environments where the depth sensor data is noisy or incomplete.
Additionally, our current implementation focuses on frame-to-frame pose
estimation with initialization from the previous frame's ground-truth
pose. In practical applications, this assumption may not hold, and
integrating our method into a full SLAM system with robust
initialization and loop closure capabilities would be necessary.
Furthermore, handling dynamic scenes and improving computational
efficiency for large-scale environments remain areas for future
exploration.

\section{Conclusion}\label{conclusion}

In this paper, we introduced \textbf{GSplatLoc}, a novel method for
ultra-precise camera localization that leverages the differentiable
rendering capabilities of 3D Gaussian splatting. By formulating pose
estimation as a gradient-based optimization problem within a fully
differentiable framework, our approach enables efficient and accurate
alignment between rendered depth maps from a pre-existing 3D Gaussian
scene and observed depth images.

Extensive experiments on the Replica and TUM RGB-D datasets demonstrate
that GSplatLoc significantly outperforms state-of-the-art SLAM systems
in terms of both translational and rotational accuracy. On the Replica
dataset, our method achieves an average Absolute Trajectory Error (ATE
RMSE) of 0.00925\,cm, surpassing existing approaches by an order of
magnitude. The method also maintains competitive performance on the TUM
RGB-D dataset, exhibiting robustness in real-world scenarios despite
challenges such as sensor noise and dynamic elements.

The superior performance of GSplatLoc can be attributed to several key
factors. The utilization of a fully differentiable depth rendering
process allows for efficient gradient-based optimization of camera
poses. The combination of depth and contour losses in our optimization
objective captures both absolute depth differences and structural
features, enhancing the accuracy of pose estimation. Moreover, employing
quaternions for rotation representation provides a continuous and
singularity-free parameter space, improving the stability and
convergence of the optimization process.

While the results are promising, there are limitations to address in
future work. The reliance on accurate depth data implies that
performance may degrade with noisy or incomplete sensor information.
Integrating GSplatLoc into a full SLAM system with robust
initialization, loop closure, and the capability to handle dynamic
scenes would enhance its applicability. Additionally, exploring methods
to improve computational efficiency for large-scale environments remains
an important direction for future research.

In conclusion, GSplatLoc represents a significant advancement in camera
localization accuracy for SLAM systems, setting a new standard for
localization techniques in dense mapping. The method's ability to
achieve ultra-precise pose estimation has substantial implications for
applications in robotics and augmented reality, where accurate and
efficient localization is critical.

\phantomsection\label{refs}
\begin{CSLReferences}{0}{1}
\bibitem[\citeproctext]{ref-kerlDenseVisualSLAM2013}
\CSLLeftMargin{1 }%
\CSLRightInline{Kerl, C., Sturm, J., Cremers, D.:
{`\href{https://doi.org/10.1109/IROS.2013.6696650}{Dense visual {SLAM}
for {RGB-D} cameras}'}, in {`2013 {IEEE}/{RSJ International Conference}
on {Intelligent Robots} and {Systems}'} (IEEE, 2013), pp. 2100--2106}

\bibitem[\citeproctext]{ref-mildenhallNeRFRepresentingScenes2022}
\CSLLeftMargin{2 }%
\CSLRightInline{Mildenhall, B., Srinivasan, P.P., Tancik, M., Barron,
J.T., Ramamoorthi, R., Ng, R.:
{`\href{https://doi.org/10.1145/3503250}{{NeRF}: Representing scenes as
neural radiance fields for view synthesis}'}\emph{Commun. ACM}, 2022,
\textbf{65}, (1), pp. 99--106. }

\bibitem[\citeproctext]{ref-sandstromPointslamDenseNeural2023}
\CSLLeftMargin{3 }%
\CSLRightInline{Sandström, E., Li, Y., Van Gool, L., Oswald, M.R.:
{`\href{http://openaccess.thecvf.com/content/ICCV2023/html/Sandstrom_Point-SLAM_Dense_Neural_Point_Cloud-based_SLAM_ICCV_2023_paper.html}{Point-slam:
{Dense} neural point cloud-based slam}'}, in {`Proceedings of the
{IEEE}/{CVF International Conference} on {Computer Vision}'} (2023), pp.
18433--18444}

\bibitem[\citeproctext]{ref-kerbl3DGaussianSplatting2023}
\CSLLeftMargin{4 }%
\CSLRightInline{Kerbl, B., Kopanas, G., Leimkühler, T., Drettakis, G.:
{`\href{https://doi.org/10.1145/3592433}{{3D Gaussian Splatting} for
{Real-Time Radiance Field Rendering}}'}\emph{ACM Transactions on
Graphics}, 2023, \textbf{42}, (4), pp. 1--14. }

\bibitem[\citeproctext]{ref-pengRTGSLAMRealtime3D2024}
\CSLLeftMargin{5 }%
\CSLRightInline{Peng, Z., Shao, T., Liu, Y., \emph{et al.}:
{`{RTG-SLAM}: {Real-time 3D Reconstruction} at {Scale} using {Gaussian
Splatting}'}, \url{http://arxiv.org/abs/2404.19706}, accessed July 2024}

\bibitem[\citeproctext]{ref-haRGBDGSICPSLAM2024}
\CSLLeftMargin{6 }%
\CSLRightInline{Ha, S., Yeon, J., Yu, H.: {`{RGBD GS-ICP SLAM}'},
\url{http://arxiv.org/abs/2403.12550}, accessed May 2024}

\bibitem[\citeproctext]{ref-yugayGaussianSLAMPhotorealisticDense2024}
\CSLLeftMargin{7 }%
\CSLRightInline{Yugay, V., Li, Y., Gevers, T., Oswald, M.R.:
{`Gaussian-{SLAM}: {Photo-realistic Dense SLAM} with {Gaussian
Splatting}'}, \url{http://arxiv.org/abs/2312.10070}, accessed June 2024}

\bibitem[\citeproctext]{ref-mur-artalOrbslam2OpensourceSlam2017}
\CSLLeftMargin{8 }%
\CSLRightInline{Mur-Artal, R., Tardós, J.D.:
{`\href{https://doi.org/10.1109/TRO.2017.2705103}{Orb-Slam2: {An}
open-source slam system for monocular, stereo, and rgb-d
cameras}'}\emph{IEEE transactions on robotics}, 2017, \textbf{33}, (5),
pp. 1255--1262. }

\bibitem[\citeproctext]{ref-camposOrbslam3AccurateOpensource2021}
\CSLLeftMargin{9 }%
\CSLRightInline{Campos, C., Elvira, R., Rodríguez, J.J.G., Montiel,
J.M., Tardós, J.D.:
{`\href{https://doi.org/10.1109/TRO.2021.3075644}{Orb-Slam3: {An}
accurate open-source library for visual, visual--inertial, and multimap
slam}'}\emph{IEEE Transactions on Robotics}, 2021, \textbf{37}, (6), pp.
1874--1890. }

\bibitem[\citeproctext]{ref-kerlRobustOdometryEstimation2013}
\CSLLeftMargin{10 }%
\CSLRightInline{Kerl, C., Sturm, J., Cremers, D.:
{`\href{https://doi.org/10.1109/ICRA.2013.6631104}{Robust odometry
estimation for {RGB-D} cameras}'}, in {`2013 {IEEE} international
conference on robotics and automation'} (IEEE, 2013), pp. 3748--3754}

\bibitem[\citeproctext]{ref-newcombeDTAMDenseTracking2011}
\CSLLeftMargin{11 }%
\CSLRightInline{Newcombe, R.A., Lovegrove, S.J., Davison, A.J.:
{`\href{https://doi.org/10.1109/ICCV.2011.6126513}{{DTAM}: {Dense}
tracking and mapping in real-time}'}, in {`2011 international conference
on computer vision'} (IEEE, 2011), pp. 2320--2327}

\bibitem[\citeproctext]{ref-whelanElasticFusionRealtimeDense2016}
\CSLLeftMargin{12 }%
\CSLRightInline{Whelan, T., Salas-Moreno, R.F., Glocker, B., Davison,
A.J., Leutenegger, S.:
{`\href{https://doi.org/10.1177/0278364916669237}{{ElasticFusion}:
{Real-time} dense {SLAM} and light source estimation}'}\emph{The
International Journal of Robotics Research}, 2016, \textbf{35}, (14),
pp. 1697--1716. }

\bibitem[\citeproctext]{ref-yen-chenInerfInvertingNeural2021}
\CSLLeftMargin{13 }%
\CSLRightInline{Yen-Chen, L., Florence, P., Barron, J.T., Rodriguez, A.,
Isola, P., Lin, T.-Y.:
{`\href{https://doi.org/10.1109/IROS51168.2021.9636708}{Inerf:
{Inverting} neural radiance fields for pose estimation}'}, in {`2021
{IEEE}/{RSJ International Conference} on {Intelligent Robots} and
{Systems} ({IROS})'} (IEEE, 2021), pp. 1323--1330}

\bibitem[\citeproctext]{ref-mullerInstantNeuralGraphics2022}
\CSLLeftMargin{14 }%
\CSLRightInline{Müller, T., Evans, A., Schied, C., Keller, A.:
{`\href{https://doi.org/10.1145/3528223.3530127}{Instant neural graphics
primitives with a multiresolution hash encoding}'}\emph{ACM Trans.
Graph.}, 2022, \textbf{41}, (4), pp. 1--15. }

\bibitem[\citeproctext]{ref-yuPlenoctreesRealtimeRendering2021}
\CSLLeftMargin{15 }%
\CSLRightInline{Yu, A., Li, R., Tancik, M., Li, H., Ng, R., Kanazawa,
A.:
{`\href{http://openaccess.thecvf.com/content/ICCV2021/html/Yu_PlenOctrees_for_Real-Time_Rendering_of_Neural_Radiance_Fields_ICCV_2021_paper.html}{Plenoctrees
for real-time rendering of neural radiance fields}'}, in {`Proceedings
of the {IEEE}/{CVF International Conference} on {Computer Vision}'}
(2021), pp. 5752--5761}

\bibitem[\citeproctext]{ref-fridovich-keilPlenoxelsRadianceFields2022}
\CSLLeftMargin{16 }%
\CSLRightInline{Fridovich-Keil, S., Yu, A., Tancik, M., Chen, Q., Recht,
B., Kanazawa, A.:
{`\href{http://openaccess.thecvf.com/content/CVPR2022/html/Fridovich-Keil_Plenoxels_Radiance_Fields_Without_Neural_Networks_CVPR_2022_paper.html}{Plenoxels:
{Radiance} fields without neural networks}'}, in {`Proceedings of the
{IEEE}/{CVF} conference on computer vision and pattern recognition'}
(2022), pp. 5501--5510}

\bibitem[\citeproctext]{ref-keethaSplaTAMSplatTrack2024}
\CSLLeftMargin{17 }%
\CSLRightInline{Keetha, N., Karhade, J., Jatavallabhula, K.M., \emph{et
al.}: {`{SplaTAM}: {Splat}, {Track} \& {Map 3D Gaussians} for {Dense
RGB-D SLAM}'}, \url{http://arxiv.org/abs/2312.02126}, accessed June
2024}

\bibitem[\citeproctext]{ref-huCGSLAMEfficientDense2024}
\CSLLeftMargin{18 }%
\CSLRightInline{Hu, J., Chen, X., Feng, B., \emph{et al.}: {`{CG-SLAM}:
{Efficient Dense RGB-D SLAM} in a {Consistent Uncertainty-aware 3D
Gaussian Field}'}, \url{http://arxiv.org/abs/2403.16095}, accessed July
2024}

\bibitem[\citeproctext]{ref-beslMethodRegistration3shapes1992}
\CSLLeftMargin{19 }%
\CSLRightInline{Besl, P.J., McKay, N.D.:
{`\href{https://doi.org/10.1117/12.57955}{Method for registration of
3-{D} shapes}'}, in {`Sensor fusion {IV}: Control paradigms and data
structures'} (Spie, 1992), pp. 586--606}

\bibitem[\citeproctext]{ref-segalGeneralizedicp2009a}
\CSLLeftMargin{20 }%
\CSLRightInline{Segal, A., Haehnel, D., Thrun, S.:
{`\href{https://doi.org/10.15607/RSS.2009.V.021}{Generalized-icp.}'}, in
{`Robotics: Science and systems'} (Seattle, WA, 2009), p. 435}

\bibitem[\citeproctext]{ref-parkColoredPointCloud2017}
\CSLLeftMargin{21 }%
\CSLRightInline{Park, J., Zhou, Q.-Y., Koltun, V.:
{`\href{http://openaccess.thecvf.com/content_iccv_2017/html/Park_Colored_Point_Cloud_ICCV_2017_paper.html}{Colored
point cloud registration revisited}'}, in {`Proceedings of the {IEEE}
international conference on computer vision'} (2017), pp. 143--152}

\bibitem[\citeproctext]{ref-steinbruckerRealtimeVisualOdometry2011}
\CSLLeftMargin{22 }%
\CSLRightInline{Steinbrücker, F., Sturm, J., Cremers, D.:
{`\href{https://doi.org/10.1109/ICCVW.2011.6130321}{Real-time visual
odometry from dense {RGB-D} images}'}, in {`2011 {IEEE} international
conference on computer vision workshops ({ICCV Workshops})'} (IEEE,
2011), pp. 719--722}

\bibitem[\citeproctext]{ref-kuipersQuaternionsRotationSequences1999}
\CSLLeftMargin{23 }%
\CSLRightInline{Kuipers, J.B.:
{`\href{https://books.google.com/books?hl=en&lr=&id=_Og9DwAAQBAJ&oi=fnd&pg=PR21&dq=kuipers+Quaternions+Rotation+Sequences+1999&ots=t3I3ky6Zut&sig=o3xQEsvNHM-t2AIySNcaxBhSc_I}{Quaternions
and rotation sequences: A primer with applications to orbits, aerospace,
and virtual reality}'} (Princeton university press, 1999)}

\bibitem[\citeproctext]{ref-zwickerEWASplatting2002}
\CSLLeftMargin{24 }%
\CSLRightInline{Zwicker, M., Pfister, H., Van Baar, J., Gross, M.:
{`\href{https://doi.org/10.1109/TVCG.2002.1021576}{{EWA}
splatting}'}\emph{IEEE Transactions on Visualization and Computer
Graphics}, 2002, \textbf{8}, (3), pp. 223--238. }

\bibitem[\citeproctext]{ref-kanopoulosDesignImageEdge1988}
\CSLLeftMargin{25 }%
\CSLRightInline{Kanopoulos, N., Vasanthavada, N., Baker, R.L.:
{`\href{https://doi.org/10.1109/4.996}{Design of an image edge detection
filter using the {Sobel} operator}'}\emph{IEEE Journal of solid-state
circuits}, 1988, \textbf{23}, (2), pp. 358--367. }

\bibitem[\citeproctext]{ref-straubReplicaDatasetDigital2019}
\CSLLeftMargin{26 }%
\CSLRightInline{Straub, J., Whelan, T., Ma, L., \emph{et al.}: {`The
{Replica Dataset}: {A Digital Replica} of {Indoor Spaces}'},
\url{http://arxiv.org/abs/1906.05797}, accessed August 2024}

\bibitem[\citeproctext]{ref-sturmBenchmarkEvaluationRGBD2012}
\CSLLeftMargin{27 }%
\CSLRightInline{Sturm, J., Engelhard, N., Endres, F., Burgard, W.,
Cremers, D.: {`\href{https://doi.org/10.1109/IROS.2012.6385773}{A
benchmark for the evaluation of {RGB-D SLAM} systems}'}, in {`2012
{IEEE}/{RSJ} international conference on intelligent robots and
systems'} (IEEE, 2012), pp. 573--580}

\bibitem[\citeproctext]{ref-sucarImapImplicitMapping2021}
\CSLLeftMargin{28 }%
\CSLRightInline{Sucar, E., Liu, S., Ortiz, J., Davison, A.J.:
{`\href{http://openaccess.thecvf.com/content/ICCV2021/html/Sucar_iMAP_Implicit_Mapping_and_Positioning_in_Real-Time_ICCV_2021_paper.html}{Imap:
{Implicit} mapping and positioning in real-time}'}, in {`Proceedings of
the {IEEE}/{CVF} international conference on computer vision'} (2021),
pp. 6229--6238}

\end{CSLReferences}
\end{document}
